Non-blocking linked data structures have been extensively studied because their distributed memory layout provides data access parallelism and scalability under high levels of contention~\cite{harris2001pragmatic,linden2013skiplist,zhang2015lockfree,michael2002high}.
To the best of the PI's knowledge, there is no existing linked data structure that provides native support for transactions.
Straightforward generic constructions can be implemented by executing all shared memory accesses in coarse-grained \emph{atomic sections}, which can employ either optimistic (e.g., STM) or pessimistic (e.g., lock inference) concurrency control.
More sophisticated approaches~\cite{bronson2010transactional,herlihy2008transactional,golan2015automatic} exploit semantic conflict detection for transaction-level synchronization to reduce benign conflicts.
However, with the specific knowledge on linked data structure, the proposed approach further optimizes the transaction execution by performing in-place conflict detection and contention management on existing nodes.

\subsection{Transactional Memory}
Initially proposed as a set of hardware extensions by Herlihy and Moss~\cite{herlihy1993transactional}, transactional memory was intended to facilitate the development of lock-free data structures.
However, due to current HTM's cache-coherency based conflict detection scheme, transactions are subject to spurious failures during page faults and context switches~\cite{dice2009early}.
This makes HTM less desirable for data structure implementations.
Considerable amount of work and ingenuity has instead gone into designing lock-free data structures using low-level synchronization primitives such as \textsc{CompareAndSwap}, which empowers researchers to devise algorithm-specific fine-grained concurrency control protocol.

Software transactional memory (STM)~\cite{shavit1997software,herlihy2003software} can be used to conveniently construct transactional data structures from their sequential counterparts: operations executed within an STM transaction are guaranteed to be transactional.
The first software transactional memory was proposed by Shavit and Touitou~\cite{shavit1997software}, which is lock-free but only supports a static set of data items.
Herlihy, et al., later presented DSTM~\cite{herlihy2003software} that supports dynamic data sets on the condition that the progress guarantee is relaxed to obstruction-freedom.
Over the years, a large number of STM implementations have been proposed~\cite{saha2006mcrt,fraser2004practical,marathe2006lowering,dice2006transactional,dalessandro2010norec}.
Complete reviews are omitted because it is out of the scope of this work.
Despite the appeal of straightforward implementation, STM based approach has yet to gain practical acceptance due to its significant runtime overhead~\cite{cascaval2008software}.
An STM instruments threads' memory accesses by recording the locations a thread reads in a \emph{read set}, and the locations it writes in a \emph{write set}. 
Conflicts are detected among the \emph{read/write sets} of different threads. 
In the presence of conflicts, only one transaction is allowed to commit while the others are aborted and restarted.
Apart from the overhead of metadata management, excessive transaction aborts in the presence of data structure ``hot-spots'' (memory locations that are constantly accessed by threads, e.g., the head node of a linked list) limit the overall concurrency~\cite{herlihy2008transactional}.
Figure~\ref{fig:stmconflict} illustrates such an example.
It shows a set implemented as an ordered linked list, where each node has two fields, an integer value and a pointer to the next node.
The initial state of the set is $\{0,3,6,9,10\}$.
Thread 1 and Thread 2 intend to insert 4 and 1, respectively.
Since these two operations commute, it is feasible to execute them concurrently~\cite{clements2015scalable}.
Nevertheless, these operations have a read/write conflict and the STM has to abort one of them. 
The inherent disadvantage of STM concurrency control is that \emph{low-level memory access conflicts do not necessarily correspond to high-level semantic conflicts}.
There is also an increasing realization that the read/write conflicts inherently provide insufficient support for concurrency when shared objects are subject to contention~\cite{koskinen2010coarse}.
It has been suggested that ``STM may not deliver the promised efficiency and simplicity for \emph{all} scenario, and multitude of approaches should be explored catering to different needs''~\cite{attiya2010inherent}.
Unlike STM-based data structures, the proposed approach eliminate false conflict and excessive aborts by adopting semantic conflict detection.

\subsection{Lock Inference}
STM implementations are typically \emph{optimistic}, which means they execute under the assumption that interferences are unlikely to occur.
They maintain computationally expensive redo- or undo-logs to allow replay or rollback in case a transaction experiences interference.
In light of this shortcoming, pessimistic alternatives based on lock inference has been proposed~\cite{mccloskey2006autolocker}.
These algorithms synthesize enough locks through static analysis to prevent data races in atomic sections.
The choice of locking granularity has an impact on the trade-off between concurrency and overhead.
Some approaches require programmers' annotation~\cite{golan2013concurrent} to specify the granularity, others automatically infer locks at a fixed granularity~\cite{emmi2007lock} or even multiple granularities~\cite{cherem2008inferring}.
Nevertheless, most approaches associate locks with memory locations, which may lead to reduced parallelism due to false conflicts as seen in STM. 
Applying these approaches to real-world programs also faces scalability changes in the presence of large libraries ~\cite{gudka2012lock} because of the high complexity involved in the static analysis process.
Applying these approaches to real-world programs also faces scalability changes in the presence of large libraries ~\cite{gudka2012lock} because of the high cyclomatic complexity~\footnote{A measure of the number of linearly independent execution paths~\cite{mccabe1976complexity}.} involved in the static analysis process.
Moreover, the use of locks degrades any non-blocking progress guarantee one might expected from using a non-blocking library.
In contrast, the proposed approach guarantees lock-free progress.

\subsection{Semantic Conflict Detection}
Considering the imprecise nature of data-based conflict detection, semantic-based approaches have been proposed to identify conflicts at a high-level (e.g., two commutative operations would not raise conflict even though they may access and modify the same memory data) which enables greater parallelism.
Because semantically independent transactions may have low-level memory access conflicts, some other concurrency control protocol must be used to protect accesses to the underlying data structure.
This results in a two-layer concurrency control.
Transactional boosting proposed by Herlihy~\cite{herlihy2008transactional} is the first dedicated treatment on building highly concurrent transactional data structures using a semantic layer of abstract locks.
The idea behind boosting is intuitive: if two operations commute they are allowed to proceed without interference (i.e., thread-level synchronization happens within the operations); otherwise they need to be synchronized at the transaction level.
It treats the base data structure as a black box and uses \emph{abstract locking} to ensure that non-commutative method calls do not occur concurrently. 
For each operation in a transaction, the boosted data structure calls the corresponding method of the underlying linearizable data structure after acquiring the abstract lock associated with that call. 
In the boosted linked list shown in Figure~\ref{fig:boosting}, the abstract lock is pertaining to each key.  
Thread 1 tries to execute \textsc{Insert(1)} and \textsc{Insert(4)} in a transaction while Thread 2 tries to execute \textsc{Remove(9)} and \textsc{Insert(4)} transactionally.
A transaction aborts when it fails to acquire an abstract lock, and it recovers from failure by invoking the inverses of already executed calls. 
Its semantic-based conflict detection approach eliminates excessive false conflicts associated with STM-based transactional data structures, but it still suffers from performance penalties due to the rollbacks of partially executed transactions.
Moreover, when applied to non-blocking data structures, the progress guarantee of the boosted data structure is degraded because of the locks used for transactional-level synchronization.

Koskinen et al.~\cite{koskinen2010coarse} later generalize the transactional boosting work and introduce a formal framework called coarse-grained transactions.
Bronson et al. proposed transaction prediction, which maps the abstract state of a set into memory blocks called predicate and relies on STM to synchronize transactional accesses~\cite{bronson2010transactional}.
Hassan et al.~\cite{hassan2014developing} proposed an optimistic version of boosting, which employs a white box design and provides throughput and usability benefits over the original boosting approach.
Other STM variants, such open nested transactions~\cite{ni2007open} support a more relaxed transaction model that leverages some semantic knowledge based on programmers' input.
The relaxation of STM systems and its implication on composability has been studied by Gramoli et al.~\cite{gramoli2013composing}.
The work by Golan-Gueta et al.~\cite{golan2015automatic} applies commutativity specification obtained from programmers' input to inferring locks for abstract data types.
